\subsection{等式约束的二次规划问题}
\label{subsec:7.2.1}
% finished

首先考虑最简单的等式约束二次规划问题, 问题形式如下
\begin{equation}
\label{eq:quadratic-programming-eq-constrained}
\begin{array}{cl}
\min & q({x}) := \frac{1}{2} {x}^T G {x} + {d}^T {x}, \\
{\rm s.t.} & A^T {x} = {b},
\end{array}
\end{equation}
其中~$A = ({a}_1, \ldots, {a}_m)^T$ 是一个~$n \times m$ 的矩阵, ${a}_i \in \mathbb{R}^n,$ ${b} \in \mathbb{R}^m.$ 假设~$A$ 列满秩, 即~$A$ 的秩为~$m \leqslant n.$ 如若不然, 只需要去掉线性相关的约束即可.

求解等式约束的二次规划问题~\eqref{eq:quadratic-programming-eq-constrained} 的一个最直接的想法是通过等约束条件消去部分变量, 将问题化为无约束极小化问题进行求解. 将矩阵$A, G,$ 向量~${d}$,
以及变量~${x}$ 进行分块
\begin{equation}
\label{eq:quadratic-programming-eq-blocks}
A = \begin{bmatrix} A_1 \\ A_2 \end{bmatrix}, ~ G = \begin{bmatrix} G_{11} & G_{12} \\ G_{21} & G_{22} \end{bmatrix}, ~ {d} = \begin{bmatrix} {d}_1 \\ {d}_2 \end{bmatrix}, ~ {x} = \begin{bmatrix} {x}_1 \\ {x}_2 \end{bmatrix},
\end{equation}
其中~$A_1$ 是可逆的~$m$ 阶方阵, ${x}_1 \in \mathbb{R}^{m},$ ${x}_2 \in \mathbb{R}^{n-m},$
矩阵~$G$ 的分块以及向量~${d}$ 的分块与~${x}$ 的分块相容.
那么等式约束的二次规划问题~\eqref{eq:quadratic-programming-eq-constrained} 中的约束方程组~$A^T {x} = {b}$ 就可写作~$A_1^T {x}_1 + A_2^T {x}_2 = {b}.$ 由于~$A_1$ 可逆, 可以利用高斯消元法求解关于~${x}_1$ 的线性方程组~$A_1^T {x}_1 = {b} - A_2^T {x}_2$ 得唯一解
\begin{equation}
\label{eq:quadratic-programming-eq-constrained-subs}
{x}_1 = A_1^{-T} ({b} - A_2^T {x}_2).
\end{equation}
将上式代入目标函数~$q({x}),$ 可得关于~${x}_2$ 的二次函数
\begin{equation}
\begin{aligned}
\psi ({x}_2) := & \frac{1}{2} {x}_2^T \left( G_{22} - G_{21} A_1^{-T} A_2^T - A_2 A_1^{-1} G_{12} + A_2 A_1^{-1} G_{12} A_1^{-T} A_2^T \right) {x}_2 \\
& + {x}_2^T \left( G_{21} - A_2 A_1^{-1} G_{11} \right) A_1^{-T} {b} + \frac{1}{2} {b}^T A_1^{-1} G_{11} A_1^{-T} {b} \\
& + {x}_2^T \left( {d}_2 - A_2 A_1^{-1} {d}_1 \right) + {d}_1^T A_1^{-T} {b}.
\end{aligned}
\end{equation}
于是, 等式约束的二次规划问题~\eqref{eq:quadratic-programming-eq-constrained} 转化为了无约束极小化问题
\begin{equation*}
\min \quad \psi({x}_2)
\end{equation*}
如果~$\nabla^2 \psi$ 正定, 解~(一阶条件) 线性方程组~$\nabla \psi ({x}_2) = 0$
可得唯一解~${x}_2^*,$ 回代入~\eqref{eq:quadratic-programming-eq-constrained-subs} 可得~${x}_1^* = A_1^{-1} ( {b} - A_2 {x}_2^* ).$ 拉格朗日乘子~${\lambda}^*$ 由~KKT 条
件~\eqref{eq:quadratic-programming-kkt} 给出, 即由
\begin{equation}
\label{eq:quadratic-programming-eq-lambda}
\begin{bmatrix} A_1 \\ A_2 \end{bmatrix} {\lambda}^* = A {\lambda}^* = -\nabla q({x}^*) = - G {x}^* - {d} = - \begin{bmatrix} G_{11} & G_{12} \\ G_{21} & G_{22} \end{bmatrix} \begin{bmatrix} {x}_1^* \\ {x}_2^* \end{bmatrix} - \begin{bmatrix} {d}_1 \\ {d}_2 \end{bmatrix}
\end{equation}
确定. 解得拉格朗日乘子的显式表达
\begin{equation*}
{\lambda}^* = -A_1^{-1} \left( G_{11} {x}_1^* + G_{12} {x}_2^* + {d}_1 \right).
\end{equation*}

\begin{exam}\label{eg:7.2.1}
用直接消元法求解下列等式约束的二次规划问题
\begin{equation*}
\begin{array}{cl}
\min & q({x}) = x_1^2 + x_2^2 + 2x_3^2 + 5x_1x_2 - 3x_2 - 7x_3, \\
{\rm s.t.} & x_1 + x_2 + x_3 = 1, \\
& x_1 - 2x_2 - 3x_3 = -2,
\end{array}
\end{equation*}
那么相应的矩阵及其划分为
\begin{equation*}
A = \left[\begin{array}{@{}cc@{}} 1 & 1 \\ 1 & -2 \\ \hdashline 1 & -3 \end{array}\right], ~~ G = \left[\begin{array}{@{}cc:c@{}} 2 & 5 & 0 \\ 5 & 2 & 0 \\ \hdashline 0 & 0 & 4 \end{array}\right], ~~ {d} = \left[\begin{array}{@{}c@{}} 0 \\ -3 \\ \hdashline -7 \end{array}\right].
\end{equation*}
将约束条件写为
\begin{equation*}
\begin{aligned}
x_1 + x_2 & = 1 - x_3, \\
x_1 - 2x_2 & = -2 + 3x_3.
\end{aligned}
\end{equation*}
利用高斯消元法, 解得~$x_1 = \frac{1}{3} x_3, ~ x_2 = 1 - \frac{4}{3} x_3.$ 代入原目标函数得
\begin{equation*}
\psi(x_3) = \frac{5 x_{3}^{2}}{3} - 4 x_{3} - 2.
\end{equation*}
$\psi$ 的~{\rm Hessian} 阵~$\nabla^2 \psi = \psi'' = \frac{10}{3} > 0,$ 于是有唯一解. 由~$0 = \nabla \psi (x_3) = \frac{10 x_{3}}{3} - 4$ 解得~$x_3^* = \frac{6}{5}.$ 回代得~$x_1^* = \frac{2}{5},$ $x_2^* = -\frac{3}{5},$ 即~${x}^* = \left( \frac{2}{5}, -\frac{3}{5}, \frac{6}{5} \right)^T.$ 最后, 解方程组~\eqref{eq:quadratic-programming-eq-lambda}, 即
\begin{equation*}
\begin{bmatrix} 1 & 1 \\ 1 & -2 \\ 1 & -3 \end{bmatrix} ~ \begin{bmatrix} \lambda_1^* \\ \lambda_2^* \end{bmatrix} = - \begin{bmatrix} 2 & 5 & 0 \\ 5 & 2 & 0 \\ 0 & 0 & 4 \end{bmatrix} \begin{bmatrix} \frac{2}{5} \\ -\frac{3}{5} \\ \frac{6}{5} \end{bmatrix} - \begin{bmatrix} 0 \\ -3 \\ -7 \end{bmatrix},
\end{equation*}
算得拉格朗日乘子~$\lambda_1^* = \frac{11}{5}, \lambda_2^* = 0.$
\end{exam}

直接消元法是一种朴素初等的方法, 非常直观, 但不是求解等式约束二次规划
问题~\eqref{eq:quadratic-programming-eq-constrained} 的最优方法. 还可以采用广义消元法~(generalized elimination) 来求解. 这个方法的名字来源于矩阵的广义逆, 利用矩阵的广义逆来表
达问题~\eqref{eq:quadratic-programming-eq-constrained} 的等式约束~(线性方程组) $A^T {x} = {b}$ 的通解, 回代入目标函数~$q({x}),$ 得到一个无约束的极小化问题. 广义消元法的本质是对变量做了线性变换, 是直接消元法~\eqref{eq:quadratic-programming-eq-constrained-subs} 的推广.

对于非齐次线性方程组~$A^T {x} = {b},$ 它如果有解, 那么它的通解可以表示为
\begin{equation}
\label{eq:general-elim-1}
{x} = Y {b} + {s}
\end{equation}
其中~$Y \in \mathbb{R}^{n\times m}$ 是系数矩阵~$A^T$ 的某个广义逆, ${s}$ 是齐次线性方程组~$A^T {x} = 0$ 的解, 或者说属于~$A$ 的列零空间(null column space). ${s}$ 也被称作可行点~${x}$ 处的可行增量. $m\times n$ 的矩阵~$A^T$ 的广义逆, 指的是一个~$n\times m$ 的矩阵~$Y,$ 满足~$A^T Y A^T = A^T.$ 同时, 非齐次线性方程组~$A^T {x} = {b}$ 有解当且仅当~$A^T Y {b} = {b}.$ 假设取定了~$n \times m$ 的矩阵~$Y$ 以及~$n \times (n-m)$ 的矩阵~$Z,$ 满足
\begin{equation}
\label{eq:eq:general-elim-req}
\begin{bmatrix} Y & Z\end{bmatrix} \text{ 非奇异, 且~} A^T Y = I_m, ~ A^T Z = 0.
\end{equation}
也就是说~$Y$ 是~$A^T$ 的一个右广义逆~(比广义逆的定义~$A^T Y A^T = A^T$ 更进一步要求~$A^T Y = I_m$). 而矩阵~$Z$ 的列~${z}_1, \ldots, {z}_{n-m}$ 构成了~$A$ 的列零空间的一组基,
也被称作既约~(reduced) 坐标方向, 矩阵~$Z$ 也被称作零空间矩阵~(null-space matrix). 矩阵~$Z^T G Z$ 被称作二次规划问题~\eqref{eq:quadratic-programming-eq-constrained} 的既约~Hessian 阵. 那么,
可行增量~${s}$ 可以表示为~${z}_1, \ldots, {z}_{n-m}$ 的线性组合
\begin{equation}
\label{eq:general-elim-2}
{s} = Z {y} = \sum\limits_{i=1}^{n-m} y_i {z}_i,
\end{equation}
其中~$y_1, \ldots, y_{n-m}$ 是每个既约坐标方向的分量, 称~${y} = (y_1, \ldots, y_{n-m})^T$ 为既约变量. 将式~\eqref{eq:general-elim-2} 代入式~\eqref{eq:general-elim-1}, 于是每一个可行点~${x}$ 都可以表示为
\begin{equation}
\label{eq:general-elim-3}
{x} = Y {b} + Z {y}.
\end{equation}
这样, 就可以用既约变量~${y}$ 代换原变量~${x}$ 将等式约束消去, 将原问题转化为了无约束的既约二次函数的极小化问题
\begin{equation}
\label{eq:general-elim-4}
\min \quad \psi({y}) := \frac{1}{2} {y}^T Z^T G Z {y} + \left( {d} + G Y {b} \right)^T Z {y} + \frac{1}{2} \left( 2{d} + G Y {b} \right)^T Y {b}.
\end{equation}

下面解无约束的极小化问题~\eqref{eq:general-elim-4}. 考虑一阶条件
\begin{equation}
\label{eq:general-elim-5}
0 = \nabla \psi ({y}) = Z^T G Z {y} + Z^T \left( {d} + G Y {b} \right).
\end{equation}
如果既约~Hessian 阵~$Z^T G Z$ 是正定的, 那么有唯一解. 解关于~${y}$ 的线性方程组
\begin{equation}
\label{eq:general-elim-y}
Z^T G Z {y} = - Z^T \left( {d} + G Y {b} \right)
\end{equation}
得唯一解~${y}^*,$ 再代入变量替换式~\eqref{eq:quadratic-programming-eq-constrained-subs} 得到原问题的唯一解~${x}^* = Y {b} + Z {y}^*.$ 由~KKT 条件~\eqref{eq:quadratic-programming-kkt}, 拉格朗日乘子~${\lambda}^*$ 满足
\begin{equation}
\label{eq:general-elim-lagrange}
A {\lambda}^* = -\left( G {x}^* + {d} \right) =: - {g}^*,
\end{equation}
两边同时左乘~$Y^T,$ 有
\begin{equation}
\label{eq:general-elim-lambda}
{\lambda}^* = \left( A^T Y \right)^T {\lambda}^* = -Y^T {g}^*.
\end{equation}

到目前为止, 还没有介绍如何选取, 或者说构造矩阵~$Y$ 和~$Z.$ 矩阵~$Y,$ $Z$ 的构造方法有很多种, 通常首选的是正交分解法~(orthogonal factorization method). 对矩阵~$A$ 进行~QR 分解, 即
\begin{equation}
\label{eq:quadratic-programming-qr-decomp-1}
A = Q \begin{bmatrix} R \\ 0 \end{bmatrix} = \begin{bmatrix} Q_1 & Q_2 \end{bmatrix} \begin{bmatrix} R \\ 0 \end{bmatrix} = Q_1 R,
\end{equation}
其中~$Q$ 是~$n \times n$ 的正交阵, $R$ 是~$m \times m$ 的可逆上三角阵. $Q_1, Q_2$ 分别是~$n \times m$ 和~$n \times (n - m)$ 的矩阵. 然后选取
\begin{equation}
\label{eq:quadratic-programming-qr-decomp-2}
Y = Q_1 R^{-T}, ~~ Z = Q_2
\end{equation}
即可满足~\eqref{eq:eq:general-elim-req} 中关于矩阵~$Y,$ $Z$ 的要求. 通过解线性方程组
\begin{equation*}
R^T {v} = {b}
\end{equation*}
得到一个特解~${v},$ 进而得到~\eqref{eq:general-elim-3} 中的特解
\begin{equation*}
Y {b} = Y R^T {v} = Q_1 {v}.
\end{equation*}
接下来只要按照广义消元法的程序, 解关于既约变量~${y}$ 的方程组~\eqref{eq:general-elim-y} 得到唯一解~${y}^*,$ 回代得到原问题最优解~${x}^* = Y {b} + Z {y}^*,$
以及用式~\eqref{eq:general-elim-lambda} 计算得到相应的拉格朗日乘子.

矩阵~$Y,$ $Z$ 的通用计算格式如下: 任选一个~$n \times (n - m)$ 的矩阵~$V$ 使得~$\begin{bmatrix} A & V \end{bmatrix}$ 是非奇异的, 且逆可以分块表示为
\begin{equation}
\label{eq:quadratic-programming-qr-decomp-general}
\begin{bmatrix} A & V \end{bmatrix}^{-1} = \begin{bmatrix} Y^T \\ Z^T \end{bmatrix},
\end{equation}
其中~$Y$ 和~$Z$ 分别是~$n \times m$ 的矩阵和~$n \times (n - m)$ 的矩阵. 依照矩阵逆的定义, 有
\begin{equation*}
I_n = \begin{bmatrix} Y^T \\ Z^T \end{bmatrix} \begin{bmatrix} A & V \end{bmatrix} = \begin{bmatrix} Y^T A & Y^T V \\ Z^T A & Z^T V \end{bmatrix},
\end{equation*}
即有~$Y^T A = I_m, Z^T A = 0,$ 满足要求~\eqref{eq:eq:general-elim-req}, 适用于广义消元法. 这也可以解释之前介绍的直接消元法, 即如果取~$V = \begin{bmatrix} 0 \\ I_{n - m} \end{bmatrix},$
同时假设矩阵~$A$ 有形如~\eqref{eq:quadratic-programming-eq-blocks} 中的分块,
那么矩阵~$\begin{bmatrix} A & V \end{bmatrix}$ 的逆可以显式地表示为
\begin{equation*}
\begin{bmatrix} Y^T \\ Z^T \end{bmatrix} = \begin{bmatrix} A_1 & 0 \\ A_2 & I_{n - m} \end{bmatrix}^{-1} = \begin{bmatrix} A_1^{-1} & 0 \\ -A_2A_1^{-1} & I_{n - m} \end{bmatrix}.
\end{equation*}
这样就得到了之前的直接消元法. 此外, 如果取~$V = Q_2,$ 其中~$Q_2$ 由矩阵~$A$ 的~QR 分解
式~\eqref{eq:quadratic-programming-qr-decomp-1} 给出, 那么由
\begin{equation*}
\begin{bmatrix} Y^T \\ Z^T \end{bmatrix} = \begin{bmatrix} A & V \end{bmatrix}^{-1} = \begin{bmatrix} Q_1R & Q_2 \end{bmatrix}^{-1} = \begin{bmatrix} R^{-1} Q_1^T \\ Q_2^T \end{bmatrix}
\end{equation*}
可得正交分解法. 尽管关于矩阵~$Y$ 和~$Z$ 有各种各样的选取方式, 然而正交分解法通常是首选的. 原因之一是计算矩阵~$Z$ 时涉及正交矩阵的操作, 而正交矩阵具有良好的数值稳定性; 其次是选取~$Z = Q_2$ 可以给出条件数~$\kappa(Z^T G Z)$ 的最优上界, 即
\begin{equation*}
\kappa(Z^T G Z) \leqslant \kappa(G).
\end{equation*}

\begin{exam}\label{eg:7.2.2}
用广义消元法~$($正交分解法$)$ 来求解例~\eqref{eg:7.2.1} 中的二次规划问题. 首先得到约束条件系数矩阵的转置矩阵~$A$ 的~{\rm QR} 分解
\begin{equation*}
A = \begin{bmatrix} 1 & 1 \\ 1 & -2 \\ 1 & -3 \end{bmatrix} = \left[\begin{array}{@{}cc:c@{}} \frac{\sqrt{3}}{3} & \frac{7 \sqrt{78}}{78} & \frac{\sqrt{26}}{26} \\ \frac{\sqrt{3}}{3} & - \frac{\sqrt{78}}{39} & - \frac{2 \sqrt{26}}{13} \\ \frac{\sqrt{3}}{3} & - \frac{5 \sqrt{78}}{78} & \frac{3 \sqrt{26}}{26} \end{array}\right] ~ \left[\begin{array}{@{}cc@{}} \sqrt{3} & - \frac{4 \sqrt{3}}{3} \\ 0 & \frac{\sqrt{78}}{3} \\ \hdashline 0 & 0 \end{array}\right]
\end{equation*}
依照式~\eqref{eq:quadratic-programming-qr-decomp-2}, 令
\begin{equation*}
Y = \left[\begin{matrix} \frac{\sqrt{3}}{3} & \frac{7 \sqrt{78}}{78} \\ \frac{\sqrt{3}}{3} & - \frac{\sqrt{78}}{39} \\ \frac{\sqrt{3}}{3} & - \frac{5 \sqrt{78}}{78} \end{matrix}\right] ~ \left[\begin{matrix} \sqrt{3} & - \frac{4 \sqrt{3}}{3} \\ 0 & \frac{\sqrt{78}}{3} \end{matrix}\right]^{-T} = \frac{1}{26} \left[\begin{matrix} 18 & 7 \\ 6 & -2 \\ 2 & -5 \end{matrix}\right], ~~ Z = \frac{1}{\sqrt{26}} \left[\begin{matrix} 1 \\ -4 \\ 3 \end{matrix}\right].
\end{equation*}
此时, $Z^T G Z = \frac{15}{13} > 0,$ 有唯一解. 代入关于既约变量~$y$ 的线性
方程组~\eqref{eq:general-elim-y} 有
\begin{equation*}
\frac{15}{13} y = - Z^T \left( d + G Y b \right) = \frac{48 \sqrt{26}}{169},
\end{equation*}
解得~$y^* = \frac{16 \sqrt{26}}{65}.$ 回代得到原问题的最优解
\begin{equation*}
{x}^* = Y {b} + Z y^* = \left[\begin{matrix} \frac{2}{5} \\ - \frac{3}{5} \\ \frac{6}{5} \end{matrix}\right],
\end{equation*}
同时, 利用式~\eqref{eq:general-elim-lambda} 得到拉格朗日乘子
\begin{equation*}
{\lambda}^* = - Y^T \left( G {x}^* + {d} \right) = \left[\begin{matrix} \frac{11}{5} \\ 0 \end{matrix}\right].
\end{equation*}
这与例~\eqref{eg:7.2.1} 中用直接消元法得到的结果是一样的.
\end{exam}

还可以直接从等式约束的二次规划问题~\eqref{eq:quadratic-programming-eq-constrained} 的~KKT 条件入手,  来寻求该问题的解. 等式约束的二次规划问题的~KKT 条件可以从式~\eqref{eq:quadratic-programming-kkt} 简化为
\begin{equation}
\label{eq:quadratic-programming-kkt-2}
\begin{aligned}
& G {x}^* + {d} + A {\lambda}^* = {0}, \\
& A^T {x}^* - {b} = {0}.
\end{aligned}
\end{equation}
写为矩阵形式, 即为
\begin{equation}
\label{eq:quadratic-programming-kkt-2-mat}
\begin{bmatrix} G & A \\ A^T & 0 \end{bmatrix} \begin{bmatrix} {x}^* \\ {\lambda}^* \end{bmatrix} = \begin{bmatrix} -{d} \\ {b} \end{bmatrix}.
\end{equation}
上式中的系数矩阵~$K := \begin{bmatrix} G & A \\ A^T & 0 \end{bmatrix}$ 被称作拉格朗日矩阵,
也被称作~KKT 矩阵, 很容易看出~KKT 矩阵是对称不定的.
 那么等式约束的二次规划问题~\eqref{eq:quadratic-programming-eq-constrained} 的求解就转化为了线性方程组~\eqref{eq:quadratic-programming-kkt-2-mat} 的求解.

如果~KKT 矩阵可逆, 将逆矩阵按相同的方式进行分块, 即
\begin{equation}
\label{eq:quadratic-programming-kkt-inverse}
\begin{bmatrix} G & A \\ A^T & 0 \end{bmatrix}^{-1} = \begin{bmatrix} C & E \\ E^T & F \end{bmatrix}
\end{equation}
那么~KKT 系统~\eqref{eq:quadratic-programming-kkt-2-mat} 的解就可写为
\begin{equation}
\label{eq:quadratic-programming-kkt-sol-1}
\begin{aligned}
{x}^* & = - C {d} + E {b}, \\
{\lambda}^* & = - E^T {d} + F {b}. \\
\end{aligned}
\end{equation}
当~$G$ 可逆时, 可以利用分块矩阵的消元法算得
\begin{equation}
\label{eq:quadratic-programming-kkt-inv-1}
\begin{aligned}
C & = G^{-1} - G^{-1} A \left( A^TG^{-1}A \right)^{-1} A^T G^{-1},\\
E & = G^{-1} A \left( A^TG^{-1}A \right)^{-1},\\
F & = - \left( A^TG^{-1}A \right)^{-1}.
\end{aligned}
\end{equation}
相关的方法被称作~Schur 补法~(Schur-complement method). 这个名字来源于, 执行分块消元操作的时候, 会得到矩阵~$G$ 的~Schur 补~$-A^TG^{-1}A.$ 要注意的是, Schur 补法要求~$G$ 可逆, 而这并不是~KKT 矩阵非奇异的必要条件; 同时基于数值稳定性等方面的考虑, 只有当~$G$ 正定以及~$A^TG^{-1}A$ 的计算开销足够小等一系列条件得到满足, Schur 补法才有应用的价值.

可以利用式~\eqref{eq:quadratic-programming-qr-decomp-general} 定义的矩阵~$Y,$ $Z$, 来求~KKT 矩阵的逆矩阵, 有
\begin{equation}
\label{eq:quadratic-programming-kkt-inv-2}
\begin{aligned}
C & = Z \left( Z^T G Z \right)^{-1} Z^T,\\
E & = Y - Z \left( Z^T G Z \right)^{-1} Z^T G Y,\\
F & = Y^T G Z \left( Z^T G Z \right)^{-1} Z^T G Y - Y^T G Y.\\
\end{aligned}
\end{equation}
这样, 可以利用矩阵~$Y,$ $Z$ 和~$Z^T G Z$ 的~Cholesky 分解~$L L^T$ 等计算方法来分解~KKT 矩阵, 相关的方法被称作零空间法~(null-space method). 这一名称来源于矩阵~$Z$ 的选取, 即~$Z$ 的列构成约束条件系数矩阵的转置矩阵~$A$ 的列零空间. 对应于不同的~$Y$ 和~$Z$ 的选取, 有不同的零空间法, 如上面介绍的广义消元法.
