\subsection*{习题}
% NOT finished

\begin{enumerate}
\item 考虑等式约束的二次规划问题
\begin{equation*}
\begin{array}{cl}
\text{minimize} & q(x) := \frac{1}{2} x^T G x + d^T x, \\
\text{subject to} & A^T x = b,
\end{array}
\end{equation*}
设$A$列满秩, $Z$为矩阵$A$的列零空间的一组基构成的矩阵, 既约Hessian阵$Z^T G Z$正定. 证明
\begin{itemize}
    \item[(1)] KKT矩阵$\begin{bmatrix} G & A \\ A^T & 0 \end{bmatrix}$非奇异;
    \item[(2)] 该二次规划问题有唯一极小点, 该点就是唯一的全局最优解.
\end{itemize}

\item 写出二次规划问题
\begin{equation*}
\begin{array}{cl}
\text{maximize} & q(x) = 6x_1 + 4x_2 - 13 - x_1^2 - x_2^2, \\
\text{subject to} & x_1 + x_2 \leqslant 3, \\
& x_1, ~ x_2 \geqslant 0
\end{array}
\end{equation*}
的KKT条件. 写出上述规划问题的对偶问题, 以及对偶问题的对偶问题.

\item SCHUR-COMPLEMENT METHOD

\item 分别利用直接消元法以及正交分解法(广义消元法)求解如下的等式约束的二次规划问题, 并检查它们得到的解是否相同.
\begin{equation*}
\begin{array}{cl}
\text{minimize} & q(x) = x_1^2 + x_2^2 + x_3^2 + 2x_3, \\
\text{subject to} & x_1 + 2x_2 - x_3 = 4, \\
& x_1 - x_2 + x_3 = -2.
\end{array}
\end{equation*}

\item 用积极集法求解下面的二次规划问题
\end{enumerate}
