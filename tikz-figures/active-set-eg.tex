\begin{figure}[H]
\centering
\larger
\begin{tikzpicture}[scale=2.4, circle defined by/.style args={center #1 and point #2}{insert path={let \p1=($(#2)-(#1)$),\n1={veclen(\x1,\y1)} in (#1) circle[radius=\n1]}}]
% \draw[gray!30, thin, dashed] (-0.5, -0.5) grid (4.6, 3.9);
\coordinate (C) at (1, 2.5);
\draw[-{Stealth}] (-0.5, 0)--(4.7, 0) node [right] {$x_1$};
\draw[-{Stealth}] (0, -0.5)--(0, 3.7) node [above] {$x_2$};
\draw[fill=gray!20] (0, 1) node (p1) [left] {$(0, 1)$} -- (2, 2) node (p2) [above] {$(2, 2)$} -- (4, 1) node (p3) [right] {$(4, 1)$} -- (2, 0) node (p4) [below] {$(2, 0)$} -- (0, 0);
\node[below = -0.1 of p4] (x01) {$\V{x}^{(0)}, \V{x}^{(1)}$};
\draw[dashed, line width=3] (2, 0) -- (1, 0) node (x23) [below] {$\V{x}^{(2)}, \V{x}^{(3)}$};
\path[name path = line1] (0, 1) -- (2, 2);
\path[name path = vertical1] (1, 0) -- (1, 4);
\draw[dashed, line width=3, name intersections = {of = line1 and vertical1, by = {x4}}]  (1, 0) -- (x4);
\node[above = -0.1 of x4] {$\V{x}^{(4)}$};
\coordinate (c1) at (0, 1);
\coordinate (c2) at (2, 2);
\coordinate (x5) at ($(c1)!(C)!(c2)$);
\draw[dashed, line width=3] (x4) -- (x5);
\node[above = -0.1 of x5] {$\V{x}^{(5)}$};
\draw[dashed, circle defined by=center C and point x5];
\coordinate (midx5) at ($(C)!0.5!(x5)$);
\draw[dashed, circle defined by=center C and point midx5];
\coordinate (p6) at ($(C)!1.5!(x5)$);
\draw[dashed, circle defined by=center C and point p6];
\begin{scope}
\clip (-.6, -.3) rectangle (2.5, 1.9);
\coordinate (p7) at (1, 0);
\draw[dashed, circle defined by=center C and point p7];
\coordinate (p8) at ($(C)!0.75!(p7)$);
\draw[dashed, circle defined by=center C and point p8];
\end{scope}
\end{tikzpicture}
\caption{积极集法求解例\ref{eg:qp-active-set-algo}的迭代示意图}
\label{fig:active-set-eg}
\end{figure}
