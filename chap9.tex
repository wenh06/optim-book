%
% Sample SBC book chapter
%
% This is a public-domain file.
%
\documentclass{SBCbookchapter}
% \usepackage[utf8]{inputenc}
\usepackage[T1]{fontenc}
\usepackage[english]{babel}
\usepackage{graphicx}
\usepackage{amsmath,amsfonts,amssymb,mathrsfs}
\usepackage{txfonts}
\usepackage{mathalfa}
\usepackage{enumitem}

\usepackage[slantfont,boldfont]{xeCJK}
\usepackage{fontspec}


%%%%%%%%%%%%%%%%%%%%%%%%%%%%%%%%%%%%%%%%%%%%%%%%%%%%%%%%%%
% set chn font
\setCJKmainfont{Noto Serif CJK SC}
\setCJKsansfont{Noto Sans CJK SC}
\setCJKmonofont{Noto Sans Mono CJK SC}
%%%%%%%%%%%%%%%%%%%%%%%%%%%%%%%%%%%%%%%%%%%%%%%%%%%%%%%%%%


%%%%%%%%%%%%%%%%%%%%%%%%%%%%%%%%%%%%%%%%%%%%%%%%%%%%%%%%%%
% custom commands
\makeatletter
\newcommand{\mylabel}[2]{#2\def\@currentlabel{#2}\label{#1}}
\makeatother

\newcommand{\R}{\mathcal{R}}
%%%%%%%%%%%%%%%%%%%%%%%%%%%%%%%%%%%%%%%%%%%%%%%%%%%%%%%%%%


\author{}
\title{特殊优化模型方法}

\begin{document}
\maketitle

\section{二次规划}
\label{sec:7.2}

具有二次目标函数和线性约束的约束优化问题被称为二次规划 (quadratic programming, QP) 问题. 这是一类非常常见的约束优化问题, 且它们也常常作为一般的约束优化问题的子问题出现. 二次规划问题的一般形式为
\begin{equation}
\label{eq:quadratic-programming}
\begin{array}{cl}
\text{minimize} & q(x) := \frac{1}{2} x^T G x + d^T x, \\
\text{subject to} & a_i^T = b_i, ~~ i \in \mathcal{E}, \\
& a_i^T x \leqslant b_i, ~~ i \in \mathcal{I},
\end{array}
\end{equation}
其中$G$是$n$阶对称方阵, $\mathcal{E}, \mathcal{I}$是两个下标的有限集, $d, a_i,$ 是$n$阶常向量, $b_i$是常数, $i \in \mathcal{E} \cup \mathcal{I},$ 约束条件有等式约束以及不等式约束. 以上形式的二次规划问题可以将约束条件以矩阵的形式表达,形成更加紧凑的形式:
\begin{equation}
\label{eq:quadratic-programming}
\begin{array}{cl}
\text{minimize} & q(x) := \frac{1}{2} x^T G x + d^T x, \\
\text{subject to} & A x \preccurlyeq b,
\end{array}
\end{equation}
其中$A$是行向量$a_i^T$排成的矩阵, $b$是常数$b_i$按相应次序排成的列向量, $i \in \mathcal{E} \cup \mathcal{I}.$


\bibliographystyle{sbc}
\bibliography{references}

\end{document}