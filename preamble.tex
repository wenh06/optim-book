%%%%%%%%%%%%%%%%%%%%%%%%%%%%%%%%%%%%%%%%%%%%%%%%%%%%%%%%%%
% packages
\usepackage[T1]{fontenc}
\usepackage[english]{babel}
\usepackage{graphicx}
\usepackage{amsmath,amsfonts,amssymb,mathrsfs}
\usepackage{txfonts}  % 可以不用
\usepackage{float}
\usepackage{multirow}
\usepackage{mathalfa}
\usepackage{enumitem}
\usepackage{bm}
\usepackage{relsize}
\usepackage[figurename=图,tablename=表]{caption}
\usepackage{chngcntr}
\counterwithin{figure}{section}
\counterwithin{table}{section}

\usepackage{arydshln}
\setlength{\dashlinegap}{2pt}
\setlength{\dashlinedash}{2pt}

% \usepackage[ruled,linesnumbered,algosection]{algorithm2e}
% \renewcommand*{\algorithmcfname}{算法}
% \renewcommand*{\algorithmautorefname}{算法}

% \makeatletter
% \renewcommand{\Indentp}[1]{%
%   \advance\leftskip by #1
%   \advance\skiptext by -#1
%   \advance\skiprule by #1}%
% \renewcommand{\Indp}{\algocf@adjustskipindent\Indentp{\algoskipindent}}
% \renewcommand{\Indpp}{\Indentp{0.5em}}%
% \renewcommand{\Indm}{\algocf@adjustskipindent\Indentp{-\algoskipindent}}
% \renewcommand{\Indmm}{\Indentp{-0.5em}}%
% \makeatother

\usepackage{algorithm,algorithmic}
\floatname{algorithm}{算法}

\usepackage{tikz}
\usetikzlibrary{trees,arrows.meta,decorations.pathmorphing,decorations.pathreplacing,shapes,shapes.geometric,backgrounds,positioning,calc,tikzmark,hobby,intersections}
% \usepackage{tkz-euclide}

\usepackage[slantfont,boldfont]{xeCJK}  % 这个只有用XeLaTeX编译才需要
%%%%%%%%%%%%%%%%%%%%%%%%%%%%%%%%%%%%%%%%%%%%%%%%%%%%%%%%%%


%%%%%%%%%%%%%%%%%%%%%%%%%%%%%%%%%%%%%%%%%%%%%%%%%%%%%%%%%%
% set chn font
% 这个在Windows下应该不需要
\setCJKmainfont{Noto Serif CJK SC}
\setCJKsansfont{Noto Sans CJK SC}
\setCJKmonofont{Noto Sans Mono CJK SC}
%%%%%%%%%%%%%%%%%%%%%%%%%%%%%%%%%%%%%%%%%%%%%%%%%%%%%%%%%%


%%%%%%%%%%%%%%%%%%%%%%%%%%%%%%%%%%%%%%%%%%%%%%%%%%%%%%%%%%
% custom commands
\makeatletter
\newcommand{\mylabel}[2]{#2\def\@currentlabel{#2}\label{#1}}
\makeatother

\newcommand{\V}[1]{{\bm{#1}}}

\DeclareMathOperator*{\argmax}{arg\,max}
\DeclareMathOperator*{\argmin}{arg\,min}

\newtheorem{thm}{定理}[section]
\newtheorem{prop}[thm]{命题}
\newtheorem{Def}[thm]{定义}
\newtheorem{lem}[thm]{引理}
\newtheorem{conj}[thm]{猜想}
\newtheorem{cor}[thm]{推论}
\newtheorem{remark}[thm]{注记}
\newtheorem{eg}[thm]{例}
\newtheorem{exam}[thm]{例}
\newtheorem{recall}[thm]{回忆}
\newtheorem{question}[thm]{问题}
\newtheorem{observation}[thm]{观察}

\newcommand{\R}{\mathcal{R}}

\numberwithin{equation}{section}

\renewcommand\arraystretch{1.3}

\allowdisplaybreaks

%%%%%%%%%%%%%%%%%%%%%%%%%%%%%%%%%%%%%%%%%%%%%%%%%%%%%%%%%%
